\documentclass[11pt, a4paper]{article}
%-----------------------
%- 	PACKAGES & SETTINGS
%-----------------------
\usepackage[a4paper,top=2cm,bottom=2cm,left=2cm,right=2cm]{geometry}
\renewcommand{\baselinestretch}{1} 
\usepackage[utf8]{inputenc}
\usepackage[italian]{babel}
\usepackage{xcolor}
\usepackage{hyperref}
\hypersetup{
    colorlinks=true,
    filecolor=magenta,      
    urlcolor=darkgray,
    linkcolor=black
}
\urlstyle{same}
\usepackage{amsmath}
\usepackage{graphicx}
\graphicspath{ {images/} }
 
%-----------------------
%- 	TITLE
%-----------------------
\title{\textbf{Relazione Progetto Sistemi Operativi}}
\author{Ludovico \textbf{Venturi}}
\coauthor{Docente di riferimento: \href{http://calvados.di.unipi.it/paragroup/torquati/}{Massimo Torquati}}
\date{UNIPI, Luglio 2020} 


%-----------------------
%- 	DOCUMENT
%-----------------------
\begin{document}
%- 	INTRO
\pagenumbering{roman} 
\maketitle
\tableofcontents
\vfill
\begin{figure}[h]
	\centering
	\includegraphics[scale=0.3]{unipi}
	\label{fig:0}
\end{figure}

\clearpage

%- 	START DOC
\pagenumbering{arabic} 
\section{Scelte progettuali}
\begin{itemize}
\item tutti i valori associati alle chiavi in un dizionario hanno lo stesso tipo
\item i tipi assumibili dai valori associati alle chiavi sono solamente interi e booleani, nello specifico: \textit{Int(..), Bool(..)}
\item non sono ammessi valori \textit{Unbound} (corrispettivo di \textit{null})
\item per semplicità di progettazione si assume che il primo valore nel dizionario, al momento della creazione, definisca il tipo dei valori del dizionario
\item la \texttt{Fold} è stata interpretata come funzione a 2 argomenti, dove il primo è l'\textit{accumulatore} che prende il valore di default del tipo del dizionario; la \texttt{Fold} è applicabile sia ad interi che a booleani
\item sono state introdotte le astrazioni funzionali multi argomento (funzioni con lista di parametri di dimensione variabile) per generalizzare dato che la \texttt{Fold} necessita di 2 argomenti
\end{itemize}
\subsection{Sintassi Astratta}
Estensione della sintassi astratta del linguaggio  \texttt{type exp = }\\
\texttt{...| CreateDict of (ide * exp)list | Insert of ide * exp * exp | Delete of  ide * exp | HasKey of ide * exp | Iterate of exp * exp	|
Fold of exp * exp | Filter of ide list * exp 	|.D..\\
...|FunCall of exp * exp list | FunArg of ide list * exp}\\\\
Estensione dei tipi esprimibili	\texttt{type evT = }\\
\texttt{...| DictVal of (ide * evT)list |...\\...| FunArgVal of ide list * exp * evT env }
\subsection{Runtime Type Checker}
Il \textit{Runtime Type Checker} è stato implementato per la creazione del dizionario, dove si verifica che tutti i valori siano consistenti nel tipo (o tutti interi, o tutti booleani). Influisce anche sull'operazione di inserimento, mentre sulle operazioni \texttt{Delete, HasKey, Filter} non ha alcuna influenza.\\ 
Per quanto riguarda le applicazioni di funzioni sui valori, ovvero \texttt{Fold, Iterate}, si demanda il type checking (già implementato nel linguaggio didattico) alle funzioni chiamate (sfruttando una sorta di \textbf{lazy evaluation} per i parametri attuali della funzione): se viene chiamata la \texttt{Fold} di una funzione che opera su booleani su di un dizionario contenente valori interi al momento della prima chiamata di funzione viene generato un \textbf{type error}.
\section{Utilizzo}
Aprire l'interprete top-level di OCaml da terminale digitando \texttt{ocaml}.\\ Importare l'interprete del linguaggio qui discusso:
\begin{center}
\texttt{\emph{\#}\# use "venturi.ml";;}
\end{center}
e da qui valutare le espressioni, entrando nel vivo del \textbf{REPL}(ReadEvalPrintLoop).\\Vari testcase sono riportati nel file \texttt{testcase.ml}. \clearpage
\subsection{Esempi}
In \texttt{venturi.ml} vengono creati 4 ambienti:\\ \\
\texttt{
emptyenv Unbound;;(*in \emph{env0}*)\\{
.. "intDict" -> [("Birman", Eint(3));("Mainecoon", Eint(13));("Siamese", Eint(17));("Foldex", Eint(21))](*in \emph{env1}*)\\
.. "boolDict" -> [("Birman", Ebool(true));("Mainecoon", Ebool(false));("Siamese", Ebool(false));("Foldex", Ebool(false))](*in \emph{env2}*)\\
.. "emptyDict" -> [] (*in \emph{env3}*)}
}\\ \\
Sempre nello stesso file vengono riportate le prove \textbf{per ogni} operazione del dizionario, incluse tutte le eccezioni da esse generabili.\\
Alcuni esempi: 
\begin{itemize}
\item \textbf{Insert} 
\begin{itemize}
\item \texttt{
eval (Insert("Ragdoll", Eint(111), Den "intDict")) env3;;}\\\textit{- : evT =
DictVal
 [("Birman", Int 3); ("Mainecoon", Int 13); ("Siamese", Int 17);
  ("Foldex", Int 21); ("Ragdoll", Int 111)]}
\item \texttt{
		eval (Insert("Siamese", Eint(111), Den "intDict")) env3;;}\\\textit{Exception: InvalidArgumentException "key already present".
}
\end{itemize}
\item \textbf{Fold}
\begin{itemize}
\item \texttt{
let env5 = bind env3 "foldFun" (eval (FunArg(["x"; "y"], Prod(Sum(Den "x", Den "y"), Eint 3) )) env3 );;\\
eval (Fold(Den "foldFun", Den "intDict")) env5;;} \\ \textit{- : evT = Int 810}

\item \texttt{
eval (Let("magazzino", CreateDict( [\\ ("mele", Eint(430)); ("banane", Eint(312)); ("arance", Eint(525)); ("pere", Eint(217))] ),\\ 
	Let("ff", FunArg(["x"; "y"], Sum(Sum(Den "x", Den "y"), Eint(1))), Fold(Den "ff", Den "magazzino")))) env3;;}\\ \textit{- : evT = Int 1488}
	
\item \texttt{eval (Fold(
			FunArg(["a"; "b"], Or(Not(Den "a"), Den "b")),\\
			Den "intDict"  
		)) env3;;} \\ \textit{Exception: Failure "Type error".}
\end{itemize}
\end{itemize}
\end{document}









